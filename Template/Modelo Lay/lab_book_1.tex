\documentclass[idxtotoc,hyperref,openany]{labbook} % 'openany' here removes the gap page between days, erase it to restore this gap; 'oneside' can also be added to remove the shift that odd pages have to the right for easier reading

\usepackage[ 
  backref=page,
  pdfpagelabels=true,
  plainpages=false,
  colorlinks=true,
  bookmarks=true,
  pdfview=FitB]{hyperref} % Required for the hyperlinks within the PDF
 
\usepackage{booktabs} % Required for the top and bottom rules in the table
\usepackage{float} % Required for specifying the exact location of a figure or table
\usepackage{graphicx} % Required for including images2
\usepackage{lipsum} % Used for inserting dummy 'Lorem ipsum' text into the template

\newcommand{\HRule}{\rule{\linewidth}{0.5mm}} % Command to make the lines in the title page
\setlength\parindent{0pt} % Removes all indentation from paragraphs

%----------------------------------------------------------------------------------------
%	DEFINITION OF EXPERIMENTS
%----------------------------------------------------------------------------------------

\newexperiment{example}{Estudo 1}
\newexperiment{example2}{Estudo 2}
\newexperiment{example3}{Estudo 3}
\newexperiment{example4}{Estudo 4}
\newexperiment{example5}{Estudo 5}

\begin{document}

%	TITLE PAGE

\frontmatter % Use Roman numerals for page numbers
\title{
\begin{center}
\HRule \\[0.4cm]
{\Huge \bfseries Resumo de Estudos \\[0.5cm] \Large Laboratório pica das galaxia}\\[0.4cm] % Degree
\HRule \\[1.5cm]
\end{center}
}
\author{\Huge Layane Tavares \\ \\ \LARGE email@layane.com \\[2cm]} % Your name and email address
\date{9 de Novembro de 2019} % Beginning date
\maketitle

\tableofcontents

\mainmatter % Use Arabic numerals for page numbers

\experiment{Estudo 1}

\lipsum[1]

\experiment{Estudo 2} % Multiple experiments can be included in a single day, this allows you to segment what was done each day into separate categories

\experiment{Estudo 3}

\lipsum[3-5]

This is a bulleted list:

\begin{itemize}
\item Item 1
\item Item 2
\item \ldots and so on
\end{itemize}

\experiment{Estudo 4}

\lipsum[6]

\experiment{Estudo 5}

\lipsum[7]


\end{document}
