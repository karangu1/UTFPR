\section{GRADUATE}

É  um estudo da AstraZeneca sobre Alzheimer. Tendo a medicação Gantenerumab ou placebo como principio ativo, após realizar a aplicação que é de injeções subcutânea o paciente fica em observação da equipe de uma a 2 horas.  Os envios são feito pela Marken até o laboratório da Covance, que por sua vez disponibiliza as caixas para envio. 

As fazes eliminatórias para o estudo inclui os testes aplicados pelas neuropsicólogas como o MMSE e FCSRT, exame de sangue, licor e de ressonância magnética. Em exames físicos o médico irá examinar olhos, orelhas,nariz, garganta,musculatura, sistema astro intestinal, geniturinário,cardiovascular,respiratório, neurológico e dermatológico. O paciente necessita de um responsável para poder ser incluso no estudo, de preferência que ele seja acompanhado principalmente em visitas onde é realizado também uma conversa com o acompanhante. É um estudo que ainda aceita pacientes para screening. 

\section{ADVANCE}

É um estudo de CIDP da Baxalta onde a medicação é a HIQVIA (hialuronidase humana combinada com a imunoglobulina),a aplicação é realizada por infusão subcutânea, primeiramente infundindo a enzima hialuronidase e depois a medicação (ou placebo). 

O paciente tem que ser maior de 18 anos e estar em tratamento estável de no minimo 6 meses. O envio é realizado pela TNT e Marken (apenas wk1- ARUP) e não disponibiliza suas próprias caixas de envio. Os pacientes levam para casa um smartphone configurado para ser usado como diário, para melhor acompanhamento.  É um estudo que ainda aceita pacientes para Screening. 

\section{CONSONANCE}

É um estudo de esclerose múltipla progressiva, realizado pela Roche A medicação é o Ocrelizumabe, os pacientes tem visitas a cada 6 meses, é feito ressonância e a coleta de sangue é realizada minutos antes de começar a infusão da pré medicação (solumedrol e antialérgico) nesse intervalo de tempo (6 meses) há contato telefônico para melhor acompanhamento do paciente. A infusão demora cerca de 4 horas e durante as consultas alguns testes são realizados como o de compreensão,processamento rápido, marcha e perguntas sobre o humor. Os pacientes tem um smartphone configurado para realizar algumas tarefas diárias referentes ao estudo (um grupo tem o diário em papel), incluindo questionários. Os envios de amostra são feitos pela Marken.  É um estudo que já esta fechado. 

\section{ENSEMBLE}

É um estudo de esclerose múltipla primária progressiva realizado pela Roche, onde a medicação é o Ocrelizumabe ou placebo, muito parecido com o CONSONANCE, realizam os mesmos testes e contém poucas diferenças. Os envios de amostra são feitos pela Marken. Não aceita mais pacientes. 

\section{THALES} 

É um estudo realizado pela AstraZeneca sobre o AVC, a medicação fornecida é o Ticagrelor e Aspirina. Nesse estudo o paciente entra com sintomas de AVC no hospital e após a avaliação médica é proposto ao paciente o tratamento do estudo, são poucas visitas (apenas acompanhamento). O estudo esta fechado opara novos participantes;

\section{EP0012}

É um estudo para epilepsia realizado pela UCB, a mediação é o Lacosamida de forma oral. Os pacientes levam um diário e tem visitas para melhor acompanhamento juntamente com contatos telefônicos. Estudo já fechado. 
